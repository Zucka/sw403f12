\chapter{Appendix}
\section{kfG}
\label{AkfG}
Listing \ref{lst:akfG} is the settings used in the program ``kfG Edit'', to be able to write ASTSableCC grammar for examples of the defined language.
This is an older iteration of the cfg, which means it does not fit into the latest one.
\begin{lstlisting}[frame=single, caption=kfG grammar for the language, label=lst:akfG]
SS           -> Blocks 
Blocks       -> Block Blocks | EPSILON 
Block        -> BeginBlock Lines EndBlock | SettingBlock
Lines        -> Line Lines | SettingBlock Lines | EPSILON 
Line         -> List | Plains eol
List         -> Numeration | Itemlist 
Numeration   -> nlist Numv1 
Numv1        -> Plains eol | Numeration 
Itemlist     -> blist Itemv1 
Itemv1       -> Plains eol | Itemlist 
BeginBlock   -> beginkwd BEBlock eol
EndBlock     -> endkwd BEBlock Eols
BEBlock      -> lcurly BEBlockv1 pipe Ident Space  rcurly
BEBlockv1    -> Ident Space | EPSILON
SettingBlock -> settingkwd lcurly Kwd colon Settingv1 Space pipe String rcurly eol
Settingv1    -> Ident| Number
Plains       -> Plain Plains | EPSILON
Plain        -> ShortBlock | CharSpace | Number
ShortBlock   -> Kwd lcurly ShortIdents pipe Plains rcurly
ShortIdents  -> ShortIdent | EPSILON 
ShortIdent   -> Kwd colon Settingv1 Space ShortIdents
String       -> CharSpace Stringv1
Stringv1     -> String  | EPSILON
CharSpace    -> char | pace
Ident        -> char Identv1
Identv1      -> Ident | EPSILON
Number       -> digit Numberv1
Numberv1     -> Number | EPSILON
Kwd          -> atsign Ident
Space        -> pace Space | EPSILON
CharAll      -> colon | char | digit | nlist | blist | scolon | percent | fslash | bslash
Eols         -> eol Eolv1
Eolv1        -> Eols | EPSILON


pace       -> ' ' 
settingkwd -> '@setting'
beginkwd   -> '@begin'
endkwd     -> '@end' 
atsign     -> @
lcurly     -> { 
rcurly     -> } 
seperator  -> <> 
pipe       -> '|' 
fslash     -> /
bslash     -> \
colon      -> : 
scolon     -> ; 
blist      -> * 
nlist      -> # 
percent    -> %\%%
eol        -> \n | \r | \r\n 
char       -> a|b|...|z|A|B|...|Z| _ |.|,
digit      -> 0|1|...|9
\end{lstlisting}


\section{CFG(SableCC)}
\label{SableCC}
Listing \ref{lst:aSableCC} is the settings used in the program ``SableCC'', to generate the lexer and parser for the language.
\begin{lstlisting}[frame=single, caption=SableCC grammer for the language, label=lst:aSableCC]
Package nisse;

Helpers
       charv1    = ['a' .. 'z'] | ['A' .. 'Z'] | '�' | '�' | '�' | '�' | '�' | '�';
       digitv1   = ['0' .. '9'] ;
       dot         = '.' ;
       comma       = ',' ;
       eolv1     = 13 10 | 13 | 10;

Tokens
       format_kwd  = '@u' | '@b' | '@i' | '@apply' | '@image' | '@title' | '@subtitle' | '@note' ;
       url         = '@url' ;
       space       =  ' ' | 9 ;
       settingkwd  = '@setting' ;
       beginkwd    = '@begin' ;
       endkwd      = '@end' ;
       atsign      = '@' ;
       lcurly      = '\{' ;
       rcurly      = '\}' ;
       pipe        = '|' ;
       fslash      = '/' ;
       bslash      = '\\' ;
       colon       = ':' ;
       scolon      = ';' ;
       blist       = '*' ;
       nlist       = '#' ;
       percent     = '%\%%' ;
       exclamation = '!' ;
       underscore  = '_' ;
       hyphen      = '-' ;
       eol         = eolv1+;
       char        = charv1+ ;
       digit       = digitv1+ ;
       float       = digitv1+ dot digitv1+ ;
       dot         = dot+;
       comma       = comma+ ;

Productions 
       nisse         = blocks* ;
       blocks        = {block} beginblock lines* endblock 
                     | {setting} space* settingblock ;
       lines         = {setting} settingblock 
                     | {numeration} numeration
                     | {itemlist} itemlist 
                     | {plaintext} plains eol ;
       numeration    = nlist numerationv1 ;
       numerationv1  = {plaintext} plains eol 
                     | {numeration} numeration;
       itemlist      = blist itemlistv1 ;
       itemlistv1    = {plaintext} plains eol  
                     | {itemlist} itemlist;
       beginblock    = beginkwd space* beblock [second]:space* eol ;
       endblock      = endkwd space* beblock [second]:space* eol ;
       beblock       = lcurly [first]:space* char [second]:space* beblockv1? rcurly ;
       beblockv1     = pipe [first]:space* char [second]:space*;
       settingblock  = settingkwd lcurly shortident pipe [second]:space* char [third]:space* rcurly space* eol;
       plains        = plainsv1+;
       plainsv1      = {shortblock} shortblock 
                     | {charall} charall;
       shortblock    = format_kwd space* lcurly shortidents? plains rcurly ;
       shortidents   = shortident+ pipe ;
       shortident    = kwd space* colon [first]:space* shortidentv1+ [second]:space*;
       shortidentv1  = {char} char 
                     | {digit} digit
                     | {float} float
                     | {colon} colon
                     | {fslash} fslash
                     | {dot} dot
                     | {underscore} underscore
                     | {hyphen} hyphen;
       kwd           = {at} atsign kwdv1+
                     | {url} url ;
       kwdv1         = {char} char
                     | {underscore} underscore;
       charall       = {colon} colon
                     | {digit} digit
                     | {float} float
                     | {semicolon} scolon
                     | {percent} percent
                     | {forwardslash} fslash
                     | {backslash} bslash
                     | {exclamation} exclamation
                     | {dot} dot 
                     | {comma} comma
                     | {char} char
                     | {space} space 
                     | {underscore} underscore
                     | {hyphen} hyphen;
\end{lstlisting}
