\chapter{Problem Formulation}
%The task of making a slideshow in applications like Microsoft�s Power Point or Apple�s Keynote is a mouse based task. How can you make a better alternative to \LaTeX~Beamer? How can you create a programming language in which a user can make a slideshow presentation without the need for a pointing device, and not have to think about the layout of single slides, but only define the general layout. Furthermore, how can you display the presentation in a way so that it will look equal on all computers. \\
%Challenges include creating a suitable domain specific language with weight on consistency and simplicity, enabling the user to focus on the content rather than the layout.

The task of making a slideshow in applications like Microsoft�s Power Point or Apple�s Keynote is a mouse based task. \\
How can you make a better alternative to \LaTeX~Beamer? \\
How can you create a programming language in which a user can make a slideshow presentation without the need for a pointing device, and not have to think about the layout of single slides, but only define the general layout? \\
Furthermore, how can you display the presentation in a way so that it will look equal on all devices? \\
A challenge regarding creating a suitable domain specific language with weight on consistency and simplicity is to enable the user to focus on the content rather than the layout.
\\ \\
The following questions provide a limitation for the scope of the project:
\\ \\
Which platform should be used for presentation? \\
Who are the users of the language? \\
Which layout decisions does the user need? \\
What are the language limitations? \\
What can be expressed in the language?