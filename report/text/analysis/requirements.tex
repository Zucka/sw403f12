\section{Requirements}
\label{LanguageRequirements}
The following requirements are set for the language to consider it as a complete language:
\\ \\
\noindent{\textbf{Capabilities}} \\
\begin{itemize}
\item It must be possible to make bullet points and numerations in the language.
\item It must be possible to import images from the Internet in the language.
\item The user should be able to change font- family, colour, size and line height.
\item The user should be able to make a transition between each slide.
\end{itemize}
\textbf{Error handling}\\
\begin{itemize}
\item The compiler should be able to tell the user where an error has occurred and what the error is.
\end{itemize}
\textbf{Test}\\
\begin{itemize}
\item The language should be easy to write in, determined by the test:
\begin{itemize}
\item The experienced user of the language should be able to make five slides, with decent content, using standard settings, within ten minutes, where the content is prewritten.
\end{itemize}
\end{itemize}

\subsection*{Limitation}
It is estimated that it can become a source of distraction if people have too many options in the language, which can lead to a decrease in slides made per minute.
\begin{itemize}
	\item Tables will not be implemented in the language due to the features complexity.
	\item Animation of text in the slideshow is not a necessity for a slideshow, which is why the feature will not be implemented.
\end{itemize}