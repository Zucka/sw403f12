\section{Parser strategy}
\subsection*{Top down}
Top down derives the parse tree from the left. This means that a non-terminal cannot have two derivations with the same start terminal or nonterminal. e.g. A -> BC | BD. This way the parser does not know which derivation to take, because both of them start with the same nonterminal.
This can be avoided by using a bottom up method.


\subsection*{Bottom up}
Bottom up derives the parse tree from the right side, which makes the expression: A -> BC | BD legal for a bottom up parser, because it looks at C and D first and then decides which derivation to pick.


\subsection*{Conclusion:}
A top down (LL) derivation of the parser will be used in this language, because that was what the group learned first, because a useful program was introduced in a course, could check if the parser was LL. This was beneficial in making the parser LL.
