\section{Parser Strategy}
\label{Parserstrategy}
There are two ways of parsing an abstract syntax tree, which is seen below:

\subsection*{Top Down}
Top down derives the parse tree from the left. This means that a nonterminal cannot have two derivations with the same start terminal or nonterminal, e.g., $A \rightarrow BC | BD$. This way the parser does not know which derivation to take, because both of them start with the same nonterminal.
This can be avoided by using a \texttt{bottom up} method.

\subsection*{Bottom Up}
Bottom up derives the parse tree from the right, which makes the expression: $A \rightarrow BC | BD$ legal for a bottom up parser, because it looks at $C$ and $D$ first and then decides which derivation to pick.

\subsection*{Conclusion}
A Bottom Up (LALR (Look Ahead Left-scan, Rightmost derivation in reverse)) derivation of the parser will be used in this language, because of the use of SableCC, which makes a LALR-parser, which can be seen in section \ref{sec:SableCC}. \\
If the lexer and parser were to be made manually, an LL parser would be preferred, because it can be visualized more easily, compared to an LR parser.