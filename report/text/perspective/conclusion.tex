\chapter{Conclusion}

The goal of this project is to make a complete programming language with a full functioning compiler, therefore during this project we have learned how to create our own programming language, and how to make a compiler for our language. \\

\noindent{The product of this project can make a decent presentation, which was intended in the beginning of the project. The requirements defined earlier in the report for the product are solved.} \\
The platform of the output (presentation) is an HTML-file, because it only requires a web browser to be able to be viewed. \\
The compiler outputs a presentation as an HTML-file, that makes for broad web browser compatibility. The user of the language does not need to have the layout of the slideshow in mind, because it is handled by the compiler itself. This creates a faster development of the slideshow for the user perspective. \\
The limitations of the product is mainly that it is very limited in how much you can design the presentation (because of its default settings). More limitations can be seen in the section about further development \ref{sec:furtherdev}. \\
The language is able to express; titles, subtitles, URLs, images and normal text. \\
More technically, the ``settings'' include; font weight, colour, size and family changes, and when to begin and end a slide.
\\ \\
From this project, and the experiences obtained through it, it can be concluded that NISSE is simpler to express slideshows in, compared to \LaTeX~Beamer, through the performed tests herein, because;
\begin{itemize}
	\item There are no packages that need to be included/downloaded to be able to use specific functionalities
	\item The language has proved to be faster to express slideshows in, because it is an HTML-based slideshow programming language, which means that all it takes to create a slideshow is a text editor and an Internet connection to send the file to the server, have it compiled and have the HTML-file sent back to the user.
\end{itemize}