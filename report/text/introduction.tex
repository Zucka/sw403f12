\chapter{Introduction}
Many people are using slideshows, in various forms, at a daily basis. There are a lot of different methods for creating a slideshow; there are programs that make it possible for the user to create slideshows relatively simple, and there are programming languages that are designed for the purpose of creating a slideshow. The slideshow solutions that will be discussed briefly are; Apple Keynote, Microsoft Office PowerPoint, \LaTeX~Beamer and the language of this report NISSE (Nearly Instantaneous Slide Show Expressions).
\\ \\
A similarity that Keynote and PowerPoint have in common is that they are both pointing device based slideshow programs, which means that without using a pointing device\footnote{A pointing device is defined as a mouse, trackball, touchpad or pointing stick} you cannot use them properly, whereas \LaTeX~Beamer is not pointing device dependent. With \LaTeX~Beamer not being a pointing device dependent slideshow programming language it makes it a direct rival to the NISSE language.
A functionality that \LaTeX~Beamer lacks out on is the option for the user to use it properly at a mobile device\footnote{A mobile device is defined as a laptop, tablet, smartphone, etc.}. With ``properly'' is meant that you can use it mobile, but \LaTeX~is a complex programming languages with a lot of packages, and the \LaTeX~Beamer class can use the same packages as Latex. \\
A goal with NISSE is to create a slideshow programming language where the user does not have to remember a lot of different packages to gain certain functionality, or have to google how to get able to use some specific functionality. When the user has finished creating the slideshow, or if he wants to compile it, he can send it to a server that automatically compiles it for him and returns the compiled slideshow in an .HTML, which makes the user able to see the slideshow.\\

%\section{Server Compiler}
%A way to make the language easier to compile is to take the workload off the user's computer, and place it on a server. This creates the aspect that you can write and compile code anywhere, on any computer, or mobile device\footnote{A mobile device is defined as a laptop, tablet, smartphone, etc.}, with internet access without any additional programs.
%Having the compilation phase on a server creates the possibility to write the slideshow\footnote{The slideshow is being compiled in to one single HTML-file} on a cellphone, which can send the code to the server and receive the presentation on the device.
%With the Internet getting faster and faster\textbf{KILDE!}, more users are online, which does not limit the possibility for the users to compile their slideshows, although the compiler can be downloaded for offline use if needed.\\

\noindent{After a walkthrough of the compiler, there will be a conclusion to summarize the report in general. Following this, the ``Further Development�� section presents ideas and functionalities which could or should be implemented in the compiler in time for the release of NISSE. Included with the report is a CD containing the developed compiler, the source code of the developed compiler and the report.}