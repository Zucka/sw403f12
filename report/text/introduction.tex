\chapter{Introduction}
Many people are using slideshows, in various forms, at a daily basis. There are a lot of different methods for creating a slideshow; there are programs that make it possible for the user to create slideshows relatively simple, and there are programming languages that are designed for the purpose of creating a slideshow. The slideshow solutions that will be discussed briefly are; Apple Keynote and Microsoft Office PowerPoint and more detailed about \LaTeX~Beamer and the language of this report NISSE (Nearly Instantaneous Slide Show Expressions).
\\ \\
A similarity that Keynote and PowerPoint have in common is that they are both pointing device\footnote{A pointing device is defined as a mouse, trackball, touchpad or pointing stick.} based slideshow programs, which means that without using a pointing device you cannot use them properly, whereas \LaTeX~Beamer is not pointing device dependent. With \LaTeX~Beamer not being a pointing device dependent slideshow programming language makes it a rival to the NISSE language.
A functionality that \LaTeX~Beamer lacks out on is the option for the user to use it properly on a mobile device\footnote{A mobile device is defined as a laptop, tablet, smartphone, etc.}. With ``properly'' it is meant that you can write it on a mobile device, but \LaTeX~is a complex programming language with a lot of packages, which also applies to \LaTeX~Beamer so the complexity is the same. \\
A goal with NISSE is to create a slideshow programming language, where the user does not have to remember a lot of different packages to gain certain functionality, or have to search on the Internet how to get some specific functionality. Then when the user wants to compile it, he can send it to a server that automatically compiles it for him and returns the compiled slideshow in a single .HTML-file, which makes the user able to see the slideshow.\\


\noindent{After a walkthrough of the compiler, there will be a conclusion to summarize the report in general. Following this, the ``Further Development��?? section presents ideas and functionalities which could be implemented in the compiler. Included with the report is a CD containing the developed compiler, the source code of the developed compiler and the report.}