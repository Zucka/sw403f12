\chapter{Parser}

In this chapter the requirements of NISSE will be presented, along with the context free grammar(CFG) of NISSE.

Requirements of the NISSE parser:
\begin{itemize}
	\item The parser should be able to receive tokens from the lexer, which it can then convert into a parse tree.
	\item It should be able to report useful error messages, if the user has written something illegal according to the CFG of NISSE.
	\item The CFG should be LL(1) in order to speed up the compiler.
	\item The CFG should also not be ambiguous, so that we can make a parser for it. Because the CFG is LL(1) (as per the previous requirement) the CFG is also not ambiguous as per the definition of LL(1).
\end{itemize}

The following is the CFG of the NISSE in EBNF:

\begin{lstlisting}[frame=single]
SS            = {Blocks} ;
Blocks        = BeginBlock {Lines} EndBlock 
              | SettingBlock ;
Lines         = SettingBlock 
              | Numeration
              | Itemlist 
              | Plains eol ;
Numeration    = nlist (Plains eol | Numeration) ;
Itemlist      = blist (Plains eol  | Itemlist) ;
BeginBlock    = beginkwd BEBlock eol ;
EndBlock      = endkwd BEBlock eol+ ;
BEBlock       = lcurly [space] char [space] [pipe char [space]] rcurly ;
SettingBlock  = settingkwd lcurly ShortIdent [space] pipe [space] char [space] rcurly eol;
Plains        = (ShortBlock | CharAll)+ ;
ShortBlock    = format_kwd lcurly [space] [ShortIdents] Plains rcurly ;
ShortIdents   = {ShortIdent}+ pipe ;
ShortIdent    = Kwd [space] colon [space] (char | digit) [space] ;
Kwd           = atsign char ;
CharAll       = colon 
              | digit 
              | scolon 
              | percent 
              | fslash 
              | bslash 
              | char 
              | space ;
\end{lstlisting}

This grammar is able to construct everything that is required of NISSE. \\
Let's look at one of the previous examples:

\begin{lstlisting}[frame=single]
Input:
@begin{fade|slide}
    Hello World
@end{slide}
\end{lstlisting}

The parse tree for this would look like: 

\begin{figure}[! h]
\centering
	 \includegraphics[width=300px]{images/ebnfexample.png}
		 \caption{Parsetree for a simple slide.}	
	\label{fig:Parsetree}
\end{figure}

%\begin{figure}[! h]
%\centering
%	 \includegraphics[width=270px]{images/Parsetreehalf(1).png}
%		 \caption{First half of parsetree for a simple slide. Using the program kfG}	
%	\label{fig:Parsetree}
%\end{figure}
%
%\begin{figure}[! h]
%\centering
%	 \includegraphics[width=270px]{images/Parsetreehalf(2).png}
%		 \caption{Second half of parsetree for a simple slide. Using the program kfG}	
%	\label{fig:Parsetree}
%\end{figure}

%The figure \ref{fig:Parsetree} is generated by the program ``kfG��, which has a slightly different structure, because each ``plain�� does not contain multiple characters. 
The figure \ref{fig:Parsetree} shows how it would parse the tree, with the input in a slide of �Hello World''. 
%The kfG structure can be seen in appendix \ref{AkfG}. We have chosen not to show anything that is more complex than the parse tree\ref{fig:Parsetree}, because the complexity would increase greatly.