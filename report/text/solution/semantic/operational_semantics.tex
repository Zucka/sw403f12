\chapter{Operational Semantics}
In this chapter some of the operational semantics of NISSE will be shown. In order to be able to read the operational semantics a few things should be known. \\
\texttt{S} denotes slides and is an environment consisting of variables (slides) pointing to a location with the \texttt{body} of the slide. \\
Also included in the environment is the keyword \texttt{next} which points to the next empty variable, and the function \texttt{new} which gets the next variable after its parameter. \\
\texttt{env} denotes an environment for settings, which also consists of variables pointing to locations containing the value of that setting. \texttt{env} does not need the \texttt{next} keyword and the \texttt{new} function because all of the settings is predefined and the only thing changed is the settings' value.
\\ \\
\noindent{$[slideshow]$}
\[ \langle S, env \rangle \ra show \]
The semantic describes how the entire slideshow is made, which are with elements in \texttt{S} that has the variables in the environment.
\\ \\ %%%%%%%
\noindent{$[specification]$}
\[ \condinfrule{\langle slide, S \rangle \ra S' \langle setting, env \rangle \ra env'} { \langle slide~setting, S, env \rangle \ra S', env'}{} \]
This semantic describes when the command slide is executed in \texttt{S}, which changes S. And when a setting is executed in the variable environment, the environment is changed.
\\ \\
\noindent{$[setting]$}
\[ \condinfrule{\langle shortident, env \rangle \ra env' \langle scope \rangle} { \langle @ setting \{ shortident | scope \}, env \rangle \ra env'}{} \]
This semantic describes the inside of a setting, how the element inside a setting block changes in the variable environment in the specific scope.
\\ \\
\noindent{$[slide]$}
\[ \condinfrule{\langle block, S[L \mapsto block] [next \mapsto new L] \rangle \ra S'} { \langle block, S \rangle \ra S'}{} \]
\[ L = S(next)\]
This semantic describes when a block is executed in \texttt{S}. This inserts the slide inside \texttt{S}, and moves the pointer to the next location where a slide can be inserted.
\\ \\
\noindent{$[block]$}
\[ \condinfrule{\langle setting, env \rangle \ra env' \langle plain,~S \rangle \ra S� \langle num \rangle \langle ilist \rangle} { \langle begin~setting~plain~num~ilist~end, S \rangle \ra S'}{} \]
This semantic shows how commands inside a block (slide) is executed, resulting in a change in a slide.
\\ \\
\noindent{$[plain]$}
\[ \condinfrule{\langle plaintext,~S \rangle \ra S� \langle shortblock,~env \rangle \ra env'} { \langle plaintext~shortblock,~S \rangle \ra S� }{} \]
This semantic illustrates how plain text is evaluated, possibly followed by a shortblock with the environment \texttt{env}.
\\ \\
\noindent{$[shortblock]$}
\[ \condinfrule{\langle formatkwd \rangle \langle shortident, env \rangle \ra env' \langle plain \rangle} { \langle formatkwd \{shortident | plain\}, env \rangle \ra env'}{} \]
This semantic changes the environment according to the \texttt{formatkwd} and \texttt{shortident}, for the specific plain text.
\\ \\
\noindent{$[shortident]$}
\[ \condinfrule{\langle env [L \mapsto V] \rangle \ra env'} { \langle kwd : V, env \rangle \ra env'}{} \]
\[ L = env(kwd)\]
This semantic changes the environment according to the keyword and the variable \texttt{V}.