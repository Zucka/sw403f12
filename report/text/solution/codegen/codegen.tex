\chapter{Code Generator}

The code generator, takes information from the semantic analysis and with it generates the output code. In the case of NISSE the output from the code generator is a mixture of HTML\footnote{Hypertext Markup Language}, CSS\footnote{Cascading Style Sheets} and Javascript. The code generator has the following requirements in order for it to generate a correct slideshow.

\subsection{Requirements}
\begin{itemize}
  \item Print out header of the HTML files, along with the neccesary libraries needed for the slideshow to function
  \item Retrieve and print out the global settings from the symbol table into the header of the HTML file
  \item Retrieve information about slide type and slide transition from the slide table about all of the slides
  \item Print out all of the plain text, numeration and item lists with the right settings, according to the symbol table
  \item Keep track of when to open and close tags, so that correct HTML is outputted, while keeping the file as small as possible
  \item End the HTML file and output the generated code to the output file.
\end{itemize}
Each of these requirements will be described further in the following subsections.

\subsection{Header file and libraries}
The common header file of all generated slides is kept in a data file called \texttt{prefix} inside the compiler. The header file consists of HTML metatags, the common CSS of all slides, the Javascript libraries and the start of the body of the HTML file. \\

The HTML metatags are there so that the slideshows work on mobile devices(i.e. Apple Iphone/Ipad or Android phones/tablets), because without them the slideshow would not scale correctly for the mobile device screens. The metatags also disables user zoom so that the user can use swipes to change slide without the need to worry about placement or size of the slide\\

The common CSS of all slides, is mainly how the page itself is setup, how a slide looks, the background color.\\

A few Javascript libraries are used to make the slideshows work, these libraries are jQuery and a plugin for jQuery called Touchswipe. jQuery is mainly used to facilitate all of the transitions of the slides, i.e. it makes it easy to animate css attributes. jQuery is also used to change CSS attributes. The plugin Touchswipe is used to make it possible to use swipe motions on mobile devices to change slide, which would otherwise not be possible. \\

A custom library is also included which uses jQuery and Touchswipe to implement the different transitions. A key hook is set up so that when the user presses a button, the code checks to see if the user pressed any of the buttons to change the slide. A similar \texttt{swipe hook} is set up using Touchswipe, to catch user swipes and checks what direction the swipe was in. These hooks then calls a common function called \texttt{go\_to\_slide} which sends the userto the requested slide. The \texttt{go\_to\_slide} function checks the \texttt{data-transition} tag of the requested slide to check which transition to run. The function then calls the appropriate function which will then change to the requested slide with the given transition. The last thing included in the custom library is so that when desktop or laptop users resize their windows, the slide and text size is automatically scaled down so that the slide still fits the whole window.\\

\subsection{Transitions}
Write about how all of the transitions "is made".

\subsection{Global settings}
The NISSE specification allows for the use of settings in a \texttt{global} scope. These settings needs to be outputted in the header of the HTML file. The code generator retrives the global settings from the symbol table, \texttt{converts} it to CSS code and outputs it into the header of the HTML file.

\subsection{End of the header}
All of the previous was all placed inside the \texttt{header} of the HTML file. The code generator now starts the \texttt{body} of the file and outputs the \texttt{wrapper} of the slides.

\subsection{Starting slides}
Slides in the HTML file are wrapped in a div that has the class \texttt{slide\_wrapper}. The div also has to have some tags set in order for the slideshow to work. The first one is the id of the div, each slide has to have a unique id equal to \texttt{slide+the slide number-1}, this is used when the slideshow moves to a slide. The second tag is the \texttt{data-transition} tag, which specifies what transition should be used when moving \texttt{away} from that slide. The last tag that has to be set is the style of the div. The first slide of the slideshow is placed in the middle of the screen and is visible, specified by the CSS \texttt{top:0\%; opacity:1;}. The rest of the slides is placed 100\% below the screen and is hidden, specified by the css \texttt{top:100\%; opacity:0;}. \\
The slide is then also wrapped in a div with the class equal to the type of slide, i.e. titleslide, dimageslide.

\subsection{Printing out plain text}
