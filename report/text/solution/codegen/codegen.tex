\chapter{Code Generator}

The code generator, takes information from the semantic analysis and with it generates the output code. In the case of NISSE the output from the code generator is a mixture of HTML\footnote{Hypertext Markup Language}, CSS\footnote{Cascading Style Sheets} and Javascript. The code generator has the following requirements in order for it to generate a correct slideshow.

\section{Requirements}
\begin{itemize}
  \item Print out header of the HTML files, along with the neccesary libraries needed for the slideshow to function
  \item Retrieve and print out the global settings from the symbol table into the header of the HTML file
  \item Retrieve information about slide type and slide transition from the slide table about all of the slides
  \item Print out all of the plain text, numeration and item lists with the right settings, according to the symbol table
  \item Keep track of when to open and close tags, so that correct HTML is outputted, while keeping the file as small as possible
  \item End the HTML file and output the generated code to the output file.
\end{itemize}
Each of these requirements will be described further in the following subsections.

\section{Header file and libraries}
The common header file of all generated slides is kept in a data file called \texttt{prefix} inside the compiler. The header file consists of HTML metatags, the common CSS of all slides, the Javascript libraries and the start of the body of the HTML file. \\

The HTML metatags are there so that the slideshows work on mobile devices(i.e. Apple Iphone/Ipad or Android phones/tablets), because without them the slideshow would not scale correctly for the mobile device screens. The metatags also disables user zoom so that the user can use swipes to change slide without the need to worry about placement or size of the slide\\

The common CSS of all slides, is mainly how the page itself is setup, how a slide looks, the background color.\\

A few Javascript libraries are used to make the slideshows work, these libraries are jQuery and a plugin for jQuery called Touchswipe. jQuery is mainly used to facilitate all of the transitions of the slides, i.e. it makes it easy to animate css attributes. jQuery is also used to change CSS attributes. The plugin Touchswipe is used to make it possible to use swipe motions on mobile devices to change slide, which would otherwise not be possible. \\

A custom library is also included which uses jQuery and Touchswipe to implement the different transitions. A key hook is set up so that when the user presses a button, the code checks to see if the user pressed any of the buttons to change the slide. A similar \texttt{swipe hook} is set up using Touchswipe, to catch user swipes and checks what direction the swipe was in. These hooks then calls a common function called \texttt{go\_to\_slide} which sends the userto the requested slide. The \texttt{go\_to\_slide} function checks the \texttt{data-transition} tag of the requested slide to check which transition to run. The function then calls the appropriate function which will then change to the requested slide with the given transition. The last thing included in the custom library is so that when desktop or laptop users resize their windows, the slide and text size is automatically scaled down so that the slide still fits the whole window.\\

\section{Transitions}
Currently the compiler supports four different types of transitions between slides. These are fade,swipe,scale and rotatescale. All of the animations required for the transitions is done through jQuery. NISSE supports using different transitions on slides in the same slideshow. The way this is done is that the first slide in the slideshow is placed in the center of the screen, while the rest of the slides are moved 100\% down to a slide stack and is hidden. The reason the rest of the slides are moved 100\% down is that there is a limitation in HTML so that if elements are hidden above other elements, the user is not able to click hyperlinks. Each transition moves the slides into place in order for the animation to work.\\
The fade transition starts by moving the slide it is switching to, to the middle of the screen(while it is still hidden). Using jQuery the CSS option opacity is then animated to 0 for the current slide. After that animation has finished the next slides opacity is then animated to 1, making it visible. Finally the previous slide is then moved back to the slide stack.\\
The swipe transition starts by moving the slide it is switching to, to either the left or right of the screen(depending if the user is moving forwards or backwards in the slides) and moves the slide up to the center of the screen. The current slide is then animated either to the left or right, while the next slide is animated into the center of the screen. While this is going on, the current slides opacity is animated to 0, while the next slides' is animated to 1. Finally the previous slide is moved back to the slide stack.\\
The scale transition starts by moving the slide it is switching to, to the middle of the screen. The scale transition makes use of the CSS option transform, which is implemented in major browsers using their own \texttt{tag}. In order to support all of the major browsers, all four versions of the transform option is animated. The four versions are \texttt{-webkit-transform} for all browsers built upon webkit( e.g. Chrome or Safari), \texttt{-moz-transform} for all of the Mozilla browsers, \texttt{-o-transform} for the Opera browser and \texttt{-ms-transform} for all of the Microsoft browsers. The option of transform used is the \texttt{scale()} option. Because jQuery can only animate numeric CSS options a workaround is used in order to animate the scale option. A CSS option that is not used in this context(in this case \texttt{border-spacing}) is animated in the range wanted, and every time the animate function steps another function is called which modifies the transform option with the value that the animate function has reached. At the same time, the next slide is faded in. Finally the previous slide is then moved back to the slide stack.\\
The rotatescale transition works similarly to the scale transition, but while also animating the scale option it also animates the rotate option of transform. Like all other transitions, the previous slide is moved back to the slide stack.\\

\section{Global settings}
The NISSE specification allows for the use of settings in a \texttt{global} scope. These settings needs to be outputted in the header of the HTML file. The code generator retrives the global settings from the symbol table, \texttt{converts} it to CSS code and outputs it into the header of the HTML file.

\section{End of the header}
All of the previous was all placed inside the \texttt{header} of the HTML file. The code generator now starts the \texttt{body} of the file and outputs the \texttt{wrapper} of the slides.

\section{Starting slides}
Slides in the HTML file are wrapped in a div that has the class \texttt{slide\_wrapper}. The div also has to have some tags set in order for the slideshow to work. The first one is the id of the div, each slide has to have a unique id equal to \texttt{slide+the slide number-1}, this is used when the slideshow moves to a slide. The second tag is the \texttt{data-transition} tag, which specifies what transition should be used when moving \texttt{away} from that slide. The last tag that has to be set is the style of the div. The first slide of the slideshow is placed in the middle of the screen and is visible, specified by the CSS \texttt{top:0\%; opacity:1;}. The rest of the slides is placed 100\% below the screen and is hidden, specified by the css \texttt{top:100\%; opacity:0;}. \\
The slide is then also wrapped in a div with the class equal to the type of slide, i.e. titleslide, dimageslide.

\section{Printing out plain text}
Because most of the work figuring out what the style of plain text should be is done in the semantic analyzer, the code generator part of printing out plaintext is relatively simple. All of the style for text can be retrieved from the symbol table. The only challenge is figuring out when to open and close tags in the HTML file. Because text has to be surrounded with tags the code generator has to check when to close tags when new text is met instead of when moving ``out'' of text. Using this method the code generator is also able to minimize the amount of code outputted by only starting a new \texttt{span} when the style of text changes. Titles and subtitles are also special in that style can change while still in the same title, which means that \texttt{span} has to be used inside titles in order for style to change inside it.

\section{Ending the HTML file}
After all of the slides has been generated, the last thing the code generator does is output some last HTML tags in order to close out the HTML file.