\chapter{Lexer}
In this chapter the requirements of the lexer for NISSE will be presented, along with a listing of the tokens for NISSE.
\section{Requirements}
Requirements for the lexer:
\begin{itemize}
		\item The lexer should be able to take any plain text file as input.
		\item The lexer should be able to recognize the input and make tokens according to the token list for NISSE.
		\item The lexer should be able to output meaningful error messages when it cannot match the input to any token.
		\item The lexer should be able to output the tokens so that the parser can use them.
\end{itemize}

\newpage
\section{Token List}
In order to create the syntax in chapter \ref{SSyntax}, tokens have to be identified for later use.
\\ \\
The tokens for the language are as follows:

\begin{lstlisting}[frame=single]
format_kwd      = '@u' | '@b' | '@i' | '@apply' | '@image' | '@title' | '@subtitle' | '@note' ;
url             = '@url' ;
space           =  ' ' ;
settingkwd      = '@setting' ;
beginkwd        = '@begin' ;
endkwd          = '@end' ;
atsign          = '@' ;
lcurly          = '{' ;
rcurly          = '}' ;
pipe            = '|' ;
fslash          = '/' ;
bslash          = '\' ;
colon           = ':' ;
scolon          = ';' ;
blist           = '*' ;
nlist           = '#' ;
percent         = 'percent' ;
exclamation     = '!' ;
eol             = eolv1 , {eolv1} ;
dot             = dotv1 , {;
comma           = commav1,{commav1} ;
\end{lstlisting}

\noindent{The abnormal in this token list is \texttt{``settingkwd''}, \texttt{``begindkwd''} and \texttt{``endkwd''}, (kwd is short for: keyword). These keywords are explicit, because they are the keywords that differ from other keywords.}
\\ \\
The reason each special keyword has its own token, is to easily see the difference between the special tokens and the normal ones, without the need to check that a given token in the syntax analyser is the right token.
\\ \\
\texttt{``Char''} is used whenever characters needs to be used in the code. Char is also used for plain text, which requires it to have the symbols' underscore, backlash, comma and period. \\
\texttt{``Digit''} is used whenever numbers need to be used in the code. \\
When numbers or characters are expected, then one such has to occur at least once.
