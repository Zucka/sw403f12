\chapter{Lexer}

In order to create the syntax in chapter \ref{SSyntax}, tokens has to be identified for later use.
\\ \\
The tokens for the language is as follows:

\begin{verbatim}
SS            = {Blocks} ;
Blocks        = BeginBlock {Lines} EndBlock 
              | SettingBlock ;
Lines         = SettingBlock 
              | Numeration
              | Itemlist 
              | Plains eol ;
Numeration    = nlist (Plains eol | Numeration) ;
Itemlist      = blist (Plains eol  | Itemlist) ;
BeginBlock    = beginkwd BEBlock eol ;
EndBlock      = endkwd BEBlock eol+ ;
BEBlock       = lcurly [space] char [space] [pipe char [space]] rcurly ;
SettingBlock  = settingkwd lcurly ShortIdent [space] pipe [space] char [space] rcurly eol;
Plains        = (ShortBlock | CharAll)+ ;
ShortBlock    = format_kwd lcurly [space] [ShortIdents] Plains rcurly ;
ShortIdents   = {ShortIdent}+ pipe ;
ShortIdent    = Kwd [space] colon [space] (char | digit) [space] ;
Kwd           = atsign char ;
CharAll       = colon 
              | digit 
              | scolon 
              | percent 
              | fslash 
              | bslash 
              | char 
              | space ;
\end{verbatim}

\noindent{The abnormal in this token list is the ``settingkwd��, ``begindkwd�� and ``endkwd�� (kwd is short for: keyword). These keywords are explicit, because they are the keywords that differs from other keywords.}
\\ \\
The reason each special keyword has it�s own token, is to easily see the difference between the special tokens and the normal ones, without the need to check that a given token in the syntax analyser it is the right token.
\\ \\
``Char�� is used whenever characters needs to used in the code. Char is also used for plain text, which require it to have the symbols underscore, backlash, comma and period. \\
``Digit�� is used whenever numbers need to be used in the code. \\
When numbers or characters are expected, then one such has to occur at least once.